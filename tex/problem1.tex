\begin{problem}[L 2.4.2]
  Use the method of exercise 2.4.2 to compute numerical solutions of the scalar initial value problem $y' = 4xy^{1/2}$, $y(0) = 1$ for $0 \leq x \leq 2$, using the steplengths $h = 0.1$, $0.05$, and $0.025$. Compare the results with the exact solution $y(x) = (1 + x^2)^2$ and deduce that the numerical solutions are not converging to the exact solution as $h \to 0$.
\end{problem}

\begin{solution}
  We are using the method
  \[
    y_{n + 2} - y_{n + 1} = \frac{h}{12}\left[4 f(x_{n + 2}, y_{n + 2}) + 8 f(x_{n + 1}, y_{n + 1}) - f(x_n, y_n)\right].
  \]
  In this case, $f(x, y) = 4xy^{1/2}$, and $x_n = nh$. We can use this to simplify the method.
  \begin{align*}
    y_{n + 2} - y_{n + 1} &= \frac{h}{12}\left[16 x_{n + 2} \sqrt{y_{n + 2}} + 32 x_{n + 1} y_{n + 1}^{1/2} - 4 x_n y_n^{1/2}\right] \\
    y_{n + 2} - y_{n + 1} &= \frac{h}{12}\left[16 (n + 2) h \sqrt{y_{n + 2}} + 32 (n + 1) h y_{n + 1}^{1/2} - 4 n h y_n^{1/2}\right] \\
    y_{n + 2} - y_{n + 1} &= \frac{4}{3} (n + 2) h^2 \sqrt{y_{n + 2}} + \frac{8}{3} (n + 1) h^2 y_{n + 1}^{1/2} - \frac{1}{3} n h^2 y_n^{1/2} \\
    \frac{4}{3} (n + 2) h^2 \sqrt{y_{n + 2}} &= y_{n + 2} + \frac{1}{3} n h^2 y_n^{1/2} - \frac{8}{3} (n + 1) h^2 y_{n + 1}^{1/2} - y_{n + 1} \\
  \end{align*}
  Let
  \begin{align*}
    A &= \frac{4}{3} (n + 2) h^2 \\
    B &= \frac{1}{3} n h^2 y_n^{1/2} - \frac{8}{3} (n + 1) h^2 y_{n + 1}^{1/2} - y_{n + 1}.
  \end{align*}
  Then we have
  \begin{align*}
    A \sqrt{y_{n + 2}} &= y_{n + 2} + B \\
    A^2 y_{n + 2} &= y_{n + 2}^2 + 2 B y_{n + 2} + B^2 \\
    0 &= y_{n+2}^2 + (2B - A^2)y_{n+2} + B^2. \\
  \end{align*}
  Solving for $y_{n + 2}$, we get
  \[
    y_{n + 2} = \frac{-C + \sqrt{C^2 - 4D}}{2}.
  \]
  The code to solve this can be found here: \url{https://github.com/capicue/6646-HW-1}. After compiling \texttt{prob1.c}, running \texttt{./prob1 0.025} will solve the problem with an $h$ value of $0.025$ and output the resulting points in a format compatible with Sage (\url{http://www.sagemath.org/}).
  
  \begin{figure}[!ht]
    \centering
    \includegraphics[scale=0.3]{../plot/plot_1_1.png}
    \includegraphics[scale=0.3]{../plot/plot_1_2.png}
    \includegraphics[scale=0.3]{../plot/plot_1_3.png}
    \caption{$h = 0.1$, $h = 0.05$, $h = 0.025$}
    \label{fig:prob1}
  \end{figure}
\end{solution}
