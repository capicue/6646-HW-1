\begin{problem}[A\&P 3.3]
  The following ODE system:
  \begin{align*}
    y_1' = \alpha - y_1 - \frac{4y_1 y_2}{1 + y_1^2}, \\
    y_2' = \beta y_1 \left(1 - \frac{y_2}{1 + y_1^2}\right),
  \end{align*}
  where $\alpha$ and $\beta$ are parameters, represents a simplified approximation to a chemical reaction. There is a parameter value $\beta_c = \frac{3\alpha}{5} - \frac{25}{\alpha}$ such that for $\beta > \beta_c$ solution trajectories decay in amplitude and spiral in phase space into a stable fixed point, whereas for $\beta < \beta_c$ trajectories oscillate without damping and are attracted to a stable limit cycle. [This is called a \emph{Hopf bifurcation}.]
  \begin{enumerate}[(a)]
    \item Set $\alpha = 10$ and use any of the discretization methods introduced in this chapter with a fixed step size $h = .01$ to approximate the solution starting at $y_1(0) = 0$, $y_2(0) = 2$, for $0 \leq t \leq 20$. Do this for the parameter values $\beta = 2$ and $\beta = 4$. For each case plot $y_1$ vs. $t$ and $y_2$ vs. $y_1$. Describe your observations.
    \item Investigate the situation closer to the critical value $\beta_c = 3.5$.
  \end{enumerate}
\end{problem}

\begin{solution}
  We use the forward Euler method for this problem since it is explicit and therefore easier to compute. Writing $x$ for $y_1$ and $y$ for $y_2$, we get
  \begin{align*}
    x_n &= x_{n-1} + h\left(\alpha - x_{n-1} - \frac{4 x_{n-1} y_{n-1}}{1 + x_{n-1}^2}\right) \\
    y_n &= y_{n-1} + h \beta x_{n-1} \left(1 - \frac{y_{n-1}}{1 + x_{n-1}^2}\right).
  \end{align*}
  We also compute the critical value for this problem when $\alpha$ is chosen to be $10$.
  \begin{align*}
    \beta_c 
    &= \frac{3\cdot 10}{5} - \frac{25}{10} \\
    &= 8.5.
  \end{align*}
  \begin{figure}[!ht]
    \centering
    \includegraphics[scale=0.3]{../plot/plot_4_1.png}
    \includegraphics[scale=0.3]{../plot/plot_4_2.png}
    \caption{$\beta = 2$: $y_1$ vs. $t$, $y_2$ vs. $y_1$}
    \label{fig:prob4_1}
  \end{figure}
  Figure \ref{fig:prob4_1} shows what happens when $\beta$ is below the critical value. Here, the solutions approach a steady limit cycle as $t \to \infty$. We can see in the first graph that $y_1$ quickly becomes periodic, and from the phase plane, we see that the solutions are quickly attracted to a cycle.
  \begin{figure}[!ht]
    \centering
    \includegraphics[scale=0.3]{../plot/plot_4_3.png}
    \includegraphics[scale=0.3]{../plot/plot_4_4.png}
    \caption{$\beta = 4$: $y_1$ vs. $t$, $y_2$ vs. $y_1$}
    \label{fig:prob4_2}
  \end{figure}
  Figure \ref{fig:prob4_2} shows what happens when $\beta$ is above the critical value. In the phase plane, we see that the solution spirals inward to a steady point as $t \to \infty$. We can also see this behavior in the graph of $y_1$ vs. $t$. In this situation, $y_1$ is still periodic, but is clearly being dampened to approach a single value.
  \begin{figure}[!ht]
    \centering
    \includegraphics[scale=0.3]{../plot/plot_4_5.png}
    \includegraphics[scale=0.3]{../plot/plot_4_6.png}
    \caption{$\beta = 3.5$: $y_1$ vs. $t$, $y_2$ vs. $y_1$}
    \label{fig:prob4_3}
  \end{figure}
  Figure \ref{fig:prob4_3} shows what happens when $\beta = \beta_c$, the critical value. At this value, the steady point asymptotic solution loses stability and turns into a stable periodic orbit. Here, the solutions spiral inward, but extremely slowly.
\end{solution}
